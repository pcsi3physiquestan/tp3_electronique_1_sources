%!TEX root = optique.tex
%----------------------------------------------------------------------------------
%----------------------------------------------------------------------------------
\documentclass[11pt,french,landscape]{article}
% Création des en-têtes et pieds de page
\RequirePackage{calc} % Evaluer des expressions mathématiques
\RequirePackage{ifthen} % Pour traiter des conditions (pour gérer le prof/eleves)
\RequirePackage{amsmath,amssymb,amsfonts,amsthm,mathtools} % Ensembles d'outils pour les expressions mathematiques
\RequirePackage[usenames,dvipsnames]{xcolor} % Utilisation des couleurs (64)
\RequirePackage[many]{tcolorbox} % Création de box et théorèmes encadrés. Charge d'autres packages
\RequirePackage{enumitem} %Gestion des liste
\RequirePackage{stmaryrd}

% Packages: pour la mise en page
\usepackage[top=2cm, bottom=2cm, left=2cm, right=2cm]{geometry} % Pour la mise en page. On pourrait la diminuer
% Packages: pour les tableaux
\usepackage{array}%Pour les largeurs de colonnes
\usepackage{booktabs} % Pour les lignes
\usepackage{multirow} % Fusion de lignes

% Packages: Mise en forme du texte
\usepackage[utf8]{inputenc} % Les fichiers doivent etre ecrits en UTF-8
\usepackage[T1]{fontenc} % Pour les césures et autre
\usepackage{lmodern} % Pack de polices
\usepackage{babel} % Prise en compte des accents et autres - Le francais est mis en en-tete
\usepackage{microtype}% Pour une précision dans l'affichage


\begin{document}
\fbox{
\begin{minipage}{0.9\textwidth}
\centering
Compétences en électrocinétique
\end{minipage}
}

\begin{tabular}{|p{15cm}|c|c|c|c|}
\toprule
 & Pas du tout & Connaissance & Cas simple & Maîtrise\\
\toprule
Acquérir correctement une tension avec une console PC & & & & \\
\midrule
Acquérir correctement une tension avec un oscilloscope (base de temps)& & & & \\
\midrule
Acquérir correctement une tension avec un oscilloscope (déviation verticale)& & & & \\
\midrule
Acquérir correctement une tension avec un oscilloscope (couplage)& & & & \\
\midrule
Acquérir correctement une tension avec un oscilloscope (mesures au curseur)& & & & \\
\midrule
Acquérir correctement une tension avec un oscilloscope (synchronisation)& & & & \\
\midrule
Mesurer correctement une tension avec un multimètre (calibre, choix du mode) & & & & \\
\midrule
Utiliser des câbles simples et coaxiaux & & & & \\
\midrule
Brancher correctement des composants R, L et C (cavaliers ou boit eà décade) & & & & \\
\midrule
Régler correctement un GBF & & & & \\
\midrule
Câbler un circuit simple sans conflit de masse & & & & \\
\midrule
Câbler un circuit avec ALI & & & & \\
\midrule
Savoir quand et comment utiliser un transformateur d'isolement & & & & \\
\midrule
Mesurer un déphasage entre deux signaux sinusoïdaux & & & & \\
\midrule
Mesurer les caractéristiques d'un signal sinusoïdal (fréquence, amplitude, valeur moyenne) & & & & \\
\midrule
Mesurer l'impédance interne d'un GBF & & & & \\
\midrule
Observer un régime transitoire correctement & & & & \\
\midrule
Mesurer les caractéristiques d'un régime transitoire d'ordre 1 (V initiale et finale, temps caractéristique) & & & & \\
\midrule
Mesurer les caractéristiques d'un régime transitoire d'ordre 2 (pseudo-période, décrément logarithmique, pulsation propre, facteur de qualité) & & & & \\
\midrule
Déterminer le type d'un filtre linéaire & & & & \\
\midrule
Mesurer un gain et une phase d'un filtre linéaire & & & & \\
\midrule
Mesurer les caractéristiques d'un filtre (f coupure, résonance, Bode, asymptote) d'un filtre linéaire & & & & \\
\bottomrule
\end{tabular}


\end{document}

